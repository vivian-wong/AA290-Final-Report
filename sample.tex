%%%%%%%%%%%%%%%%%%%%%%%%%%%%%%%%%%%%%%%%%%%%%%%%%%%%%%%%%%%%%%%%%%%%%%%%%%%%%%%%
%2345678901234567890123456789012345678901234567890123456789012345678901234567890
%        1         2         3         4         5         6         7         8

\documentclass[letterpaper, 10 pt, conference]{ieeeconf}  % Comment this line out
                                                          % if you need a4paper
%\documentclass[a4paper, 10pt, conference]{ieeeconf}      % Use this line for a4
                                                          % paper

\IEEEoverridecommandlockouts                              % This command is only
                                                          % needed if you want to
                                                          % use the \thanks command
\overrideIEEEmargins
% See the \addtolength command later in the file to balance the column lengths
% on the last page of the document



% The following packages can be found on http:\\www.ctan.org
%\usepackage{graphics} % for pdf, bitmapped graphics files
%\usepackage{epsfig} % for postscript graphics files
%\usepackage{mathptmx} % assumes new font selection scheme installed
%\usepackage{times} % assumes new font selection scheme installed
%\usepackage{amsmath} % assumes amsmath package installed
%\usepackage{amssymb}  % assumes amsmath package installed

\title{\LARGE \bf
Title of your project
}

%\author{ \parbox{3 in}{\centering Huibert Kwakernaak*
%         \thanks{*Use the $\backslash$thanks command to put information here}\\
%         Faculty of Electrical Engineering, Mathematics and Computer Science\\
%         University of Twente\\
%         7500 AE Enschede, The Netherlands\\
%         {\tt\small h.kwakernaak@autsubmit.com}}
%         \hspace*{ 0.5 in}
%         \parbox{3 in}{ \centering Pradeep Misra**
%         \thanks{**The footnote marks may be inserted manually}\\
%        Department of Electrical Engineering \\
%         Wright State University\\
%         Dayton, OH 45435, USA\\
%         {\tt\small pmisra@cs.wright.edu}}
%}

\author{Your name% <-this % stops a space
\thanks{Your affiliation and email:
        {\tt\small name@stanford.edu}}%
}


\begin{document}



\maketitle
\thispagestyle{empty}
\pagestyle{empty}


%%%%%%%%%%%%%%%%%%%%%%%%%%%%%%%%%%%%%%%%%%%%%%%%%%%%%%%%%%%%%%%%%%%%%%%%%%%%%%%%
\begin{abstract}

Short description of your AA 290 project.

\end{abstract}


%%%%%%%%%%%%%%%%%%%%%%%%%%%%%%%%%%%%%%%%%%%%%%%%%%%%%%%%%%%%%%%%%%%%%%%%%%%%%%%%
\section{Introduction}

State here the general topic of your AA 290 project.

\subsection{Literature}

Provide here a thorough literature review for your problem (about 10-12 papers).  I would like to see your judgment about each paper you cite (e.g., this paper does not address in an efficient way the case with...).

\subsection{Motivation of proposed work} 
In light of the previous literature review, explain why your project is important.

\subsection{Statement of work} 
State the objectives of your project at a general level.




%%%%%%%%%%%%%%%%%%%%%%%%%%%%%%%%%%%%%%%%%%%%%%%%%%%%%%%%%%%%%%%%%%%%%%%%%%%%%%%%
\section{Problem formulation}
Provide here a sound formulation of your problem. Also, describe here your objectives in a formal way.

\section{Proposed Solution}

\section{Simulation/Experiments}

\addtolength{\textheight}{-3cm}   % This command serves to balance the column lengths
                                  % on the last page of the document manually. It shortens
                                  % the textheight of the last page by a suitable amount.
                                  % This command does not take effect until the next page
                                  % so it should come on the page before the last. Make
                                  % sure that you do not shorten the textheight too much.

%%%%%%%%%%%%%%%%%%%%%%%%%%%%%%%%%%%%%%%%%%%%%%%%%%%%%%%%%%%%%%%%%%%%%%%%%%%%%%%%
%%%%%%%%%%%%%%%%%%%%%%%%%%%%%%%%%%%%%%%%%%%%%%%%%%%%%%%%%%%%%%%%%%%%%%%%%%%%%%%%
\section{Conclusions}

\subsection{Conclusions}

\bibliographystyle{unsrt} 
\bibliography{Biblio}  

\section*{Appendix --- style guidelines}

\subsection{Figures and Tables}

Position figures and tables at the tops and bottoms of columns.
Avoid placing them in the middle of columns. Large figures and tables
may span across both columns. Figure captions should be below the figures;
 table captions should be above the tables. Avoid placing figures and tables
  before their first mention in the text. Use the abbreviation ``Fig. 1'',
  even at the beginning of a sentence.
Figure axis labels are often a source of confusion.
Try to use words rather then symbols. As an example write the quantity ``Inductance",
 or ``Inductance L'', not just.
 Put units in parentheses. Do not label axes only with units.
 In the example, write ``Inductance (mH)'', or ``Inductance L (mH)'', not just ``mH''.
 Do not label axes with the ratio of quantities and units.
 For example, write ``Temperature (K)'', not ``Temperature/K''.

\subsection{Numbering}

Number reference citations consecutively in square brackets \cite{Garcia.ea:Auto89}.
 The sentence punctuation follows the brackets \cite{Garcia.ea:Auto89}.
 Refer simply to the reference number, as in \cite{Garcia.ea:Auto89}.
 Do not use ``ref. \cite{Garcia.ea:Auto89}'' or ``reference \cite{Garcia.ea:Auto89}''.
Number footnotes separately in superscripts\footnote{This is a footnote}
Place the actual footnote at the bottom of the column in which it is cited.
Do not put footnotes in the reference list.
Use letters for table footnotes (see Table I).

\subsection{Abbreviations and Acronyms}

Define abbreviations and acronyms the first time they are used in the text,
even after they have been defined in the abstract. Abbreviations such as
IEEE, SI, CGS, ac, dc, and rms do not have to be defined. Do not use
abbreviations in the title unless they are unavoidable.

\subsection{Equations}

Number equations consecutively with equation numbers in parentheses flush
 with the right margin, as in (1). To make your equations more compact
 you may use the solidus (/), the exp. function, or appropriate exponents.
  Italicize Roman symbols for quantities and variables, but not Greek symbols.
   Use a long dash rather then hyphen for a minus sign. Use parentheses to avoid
    ambiguities in the denominator.
Punctuate equations with commas or periods when they are part of a sentence:
$$\Gamma_2 a^2 + \Gamma_3 a^3 + \Gamma_4 a^4 + ... = \lambda \Lambda(x),$$
where $\lambda$ is an auxiliary parameter.

Be sure that the symbols in your equation have been defined before the
equation appears or immediately following.
Use ``(1),'' not ``Eq. (1)'' or ``Equation (1),''
except at the beginning of a sentence: ``Equation (1) is ...''.

%\begin{table}
%\caption{An Example of a Table}
%\label{table_example}
%\begin{center}
%\begin{tabular}{|c||c|}
%\hline
%One & Two\\
%\hline
%Three & Four\\
%\hline
%\end{tabular}
%\end{center}
%\end{table}

   \begin{figure}[thpb]
      \centering
      %\includegraphics[scale=1.0]{figurefile.jpg}
      \caption{Inductance of oscillation winding on amorphous
       magnetic core versus DC bias magnetic field}
      \label{figurelabel}
   \end{figure}




\end{document}
